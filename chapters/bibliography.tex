\begin{thebibliography}{9}

% \addcontentsline{toc}{chapter}{thebibliography}

\bibitem{BOM}
Yan Wang, Ruiqi Liu, Tong Gao, Feng Shu, Xuemei Lei, Yongpeng Wu, Guan Gui, Jiangzhou Wang,
 "A Novel RFID Authentication Protocol Based on A Block-Order-Modulus Variable Matrix Encryption Algorithm", 
  \url{https://doi.org/10.48550/arXiv.2312.10593}

\bibitem{Vaudenay}
 Vaudenay, S. (2007). 
 "On Privacy Models for RFID".  
 In: Kurosawa, K. (eds) Advances in Cryptology - ASIACRYPT 2007. ASIACRYPT 2007. 
 Lecture Notes in Computer Science, vol 4833. Springer, Berlin, Heidelberg. 
 \url{https://doi.org/10.1007/978-3-540-76900-2_5}

\bibitem{Tiplea}
F. L. Țiplea, “Lessons to be learned for a good design of private RFID schemes,” IEEE Transactions on Dependable and Secure Computing,
vol. 19, no. 4, pp. 2384-2395, 2022.

\bibitem{Shoup}
Shoup, Victor. "Sequences of games: a tool for taming complexity in security proofs." cryptology eprint archive (2004).

\bibitem{Bocchetti}
Bocchetti, Salvatore. "Security and privacy in rfid protocols." July (2006) - Citeseer.

\bibitem{Domdouzis}
Domdouzis, Konstantinos, Bimal Kumar, and Chimay Anumba. "Radio-Frequency Identification (RFID) applications: A brief introduction." 
Advanced Engineering Informatics 21.4 (2007): 350-355.

\bibitem{Feldhofer}
Feldhofer, M., Rechberger, C. (2006). "A Case Against Currently Used Hash Functions in RFID Protocols". 
In: Meersman, R., Tari, Z., Herrero, P. (eds) On the Move to Meaningful Internet Systems 2006: OTM 2006 Workshops. OTM 2006. 
Lecture Notes in Computer Science, vol 4277. Springer, Berlin, Heidelberg. 
\url{https://doi.org/10.1007/11915034_61}

\bibitem{Robshaw}
Robshaw, M.J.B. (2006). "Searching for Compact Algorithms: cgen". 
In: Nguyen, P.Q. (eds) Progress in Cryptology - VIETCRYPT 2006. VIETCRYPT 2006. Lecture Notes in Computer Science, vol 4341. 
Springer, Berlin, Heidelberg. 
\url{https://doi.org/10.1007/11958239_3}

\bibitem{Feldhofer2}
Feldhofer, M., Dominikus, S., Wolkerstorfer, J. (2004). 
"Strong Authentication for RFID Systems Using the AES Algorithm". 
In: Joye, M., Quisquater, JJ. (eds) Cryptographic Hardware and Embedded Systems - CHES 2004. CHES 2004. Lecture Notes in Computer Science, 
vol 3156. Springer, Berlin, Heidelberg. \url{https://doi.org/10.1007/978-3-540-28632-5_26}

\bibitem{Impossibility_results}
Armknecht, F., Sadeghi, AR., Scafuro, A., Visconti, I., Wachsmann, C. (2010). 
"Impossibility Results for RFID Privacy Notions". In: Gavrilova, M.L., Tan, C.J.K., Moreno, E.D. (eds) Transactions on Computational Science XI. 
Lecture Notes in Computer Science, vol 6480. Springer, Berlin, Heidelberg. \url{https://doi.org/10.1007/978-3-642-17697-5_3}

\bibitem{PV model}
Radu-Ioan Paise and Serge Vaudenay. 2008. Mutual authentication in RFID: security and privacy. In Proceedings of the 2008 ACM symposium on 
Information, computer and communications security (ASIACCS '08). Association for Computing Machinery, New York, NY, USA, 292-299. 
\url{https://doi.org/10.1145/1368310.1368352}

\bibitem{BAN logic}
Wessels, Jan, and C. F. Bv. "Application of BAN-logic." CMG FINANCE BV 19 (2001) - ipa.win.tue.nl : 1-23. 

\bibitem{PUFs}
Hristea, C., Ţiplea, F.L. (2023). "Destructive Privacy and Mutual Authentication in Vaudenay's RFID Model". 
In: Balas, V.E., Jain, L.C., Balas, M.M., Baleanu, D. (eds) Soft Computing Applications. SOFA 2020. 
Advances in Intelligent Systems and Computing, vol 1438. Springer, Cham. \url{https://doi.org/10.1007/978-3-031-23636-5_51}

\end{thebibliography}