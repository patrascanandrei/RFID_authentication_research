\chapter*{Motivation} 
\addcontentsline{toc}{chapter}{Motivation}

The choice towards studying RFID schemes and their security came for multiple reasons. 

This technology is widely adopted in various industries and in a continuous advancement, meaning it is of prime importance
that secure and private protocols are implemented for it. The applications for this fascinating technology are vast, they provide 
near-instant authentication for multiple tags at the same time, meaning it can be used in the medical field, for supply chains, in 
the automotive and construction industry and where it all started, in a military context. They provide access control from authentication 
of staff to providing access to ski lifts. With this rapid deployment of this technology, multiple issues have emerged. 
Solid protocols for RFID would help in establishing a productive and secure environment.

A typical RFID scheme is in a context where achieving security properties is difficult. The tags come with multiple 
constraints like processing power and limited memory and some of them can only work when interacting with a reader. This means 
that the RFID protocols need to strike a balance between lightness of operations and a solid security parameter. This balancing act means
that mature and proven cryptographic primitives are not always available, which sparked my interest in this study domain.
The tags are also cheap devices that are not tamper-proof, meaning new difficulties in developing solid protocols concerning how
memory is used.

Lastly the development of RFID schemes often primarily focus on the security of the schemes while the privacy takes a secondary role, this
also motivated the creation of this work.

All these limitations and the possible applications and directions of study of RFID schemes lead to a challenging but rewarding 
field of study.