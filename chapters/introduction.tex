\chapter*{Introduction} 
\addcontentsline{toc}{chapter}{Introduction}

    In a world that grows more and more a reliability on smart devices the need for secure and private protocols grows with it. One 
such technology that makes this possible is RFID. RFID is the acronym of \textbf{Radio Frequency Identification} and it designates
a wireless technology that uses radio signals aiming to identify objects and persons by the means of devices (tags) attached to them.

    During World-War II, the British wanted to distinguish between enemy and their own returning aircraft. To achieve this they placed transponders
on their aircraft which would respond to queries from the base stations. This was called the Identity Friend or Foe (IFF) system, which is considered to be the
first use of RFID.

    Since then RFID technology has been applied to a number of technical and scientific fields. In medicine RFID tags are used in blood 
transfusion and analysis. Patients can wear a RFID tag attached to a wristband that contains information about their medical attributes. 
A RFID reader scans the tags attached to blood bags and finds the appropriate one for a specific patient. 

In the aeronautics industry, RFID tags are used for the supply chain. Boeing, for instance, tags ship crates containing aeronautical
components. An advance report is sent to the depots regarding the contents of these tags and the information is later verified on delivery.

In the automotive industry, the tags can be attached to parts of a car and tracked during assembly. Each customer has their own preferences
for their car and RFID tags can be used to track components and avoid incorrect placements. 

RFID also has many applications in the construction industry \cite{Domdouzis}, from automated tracking of pipe spools to tracking the 
location of buried assets. The tags can serve the purpose of identifying if a valve is the correct one, at the correct pressure and 
in the correct location. 

This work covers a mutual identification RFID scheme and studies primarily the privacy property of it. The contribution of this thesis
consists of presenting how an adversary would interact and break privacy in a multitude of circumstances. These conditions are determined
by the choice and capabilities of the used tags. 

The structure is as follows: the first chapter presents the necessary theoretical material about RFID schemes and classes of adversaries.
The second chapter covers the protocol to be studied. The third chapter shows how an attacker interacts with a scheme for different 
classes of tags and provides the theoretical proofs. Lastly the forth chapter presents the case study on the attacks from chapter 3 
for the block-order-modulus protocol, the effects on privacy and protocol development. It shows how the claimed security properties do 
not hold in the context of tag corruption. 