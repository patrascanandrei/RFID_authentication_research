\chapter*{Introduction} 
\addcontentsline{toc}{chapter}{Introduction}

    In a world that grows more and more a reliability on smart devices the need for secure and private protocols grows with it. One 
such technology that makes this possible is RFID. RFID is the acronym of \textbf{Radio Frequency Identification} and it designates
a wireless technology that uses radio signals aiming to identify objects and persons by the means of devices (tags) attached to them.

    During World-War II, the British wanted to distinguish between enemy and their own returning aircraft. To achieve this they placed transponders
on their aircraft which would respond to queries from the base stations. This was called the Identity Friend or Foe (IFF) system, which is considered to be the
first use of RFID.

    Since then RFID technology has been applied to a number of technical and scientific fields. In medicine RFID tags are used in blood transfusion and analysis. 
A RFID reader scans the tags attached to blood bags and finds the appropriate one for a specific patient. In the aeronautics industry, 
RFID tags are used for the supply chain. In the automotive industry, the tags can be attached to parts of a car and tracked during assembly.
RFID also has many applications in the construction industry \cite{Domdouzis}, from automated tracking of pipe spools to tracking the 
location of buried assets.