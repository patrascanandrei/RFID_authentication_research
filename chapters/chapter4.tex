\chapter{Case study}

    For the authentication proposed in \cite{BOM}, the tag stores in it's permanent memory the values: \textit{A, B, p, f(p), S, Z, $W_i$}. The global temporary memory will hold the value $N_t$
    used to authenticate the reader. It is generated at step 2 and held until just after step 10.

    \definecolor{mygray}{gray}{0.75}

    \begin{adjustwidth}{-60pt}{}
    \procedureblock[colspace=-0.5cm]{The two way DBKM-SUEO-SUM-RFID protocol: use of temporary variable $N_t$}{
    \textbf{Reader} \< \< \textbf{Tag} \\[-1ex]
    \textit{A, B, p, f(p),$N_r$, S, Z, $W_i$} \< \< \textit{A, B, p, f(p),$N_t$, S, Z, $W_i$} \\[-1.5ex]
    % \\
    \< \< \text{ \scriptsize 2.Generates \colorbox{mygray}{$N_t$}} \\[-1ex]
    \< \< \text{ \scriptsize E($N_t||S$,A,p)} \\[-4ex]
    % \\
    \< \sendmessageleft{top=\text{ \scriptsize 3.E($N_t||S$,A,p)}} \< \\[-4ex]
    % \\
    \text{ \scriptsize 4.D(E(\colorbox{mygray}{$N_t$}$||$\colorbox{olive}{$S$},A,p),B,p)} \< \< \\[-1ex]
    \text{ \scriptsize Generates $N_r$} \< \< \\[-1ex]
    \text{ \scriptsize E($N_r||S$,A,p)} \< \< \\[-1ex]
    % \\
    \text{ \scriptsize ...} \< \< \\[-1ex]
    % \\
    \text{ \scriptsize 8.D(E($N_r||S_d||S_p||S_c$,A,p),B,p)} \< \< \\[-1ex]
    \text{ \scriptsize E(\colorbox{mygray}{$N_t$}$||S_d||S_p||S_c$, A, p)} \< \< \\[-4ex]
    % \\
    \<\sendmessageright{top = \text{ \scriptsize 9.E($N_t||S_d||S_p||S_c$,A,p)}}\<\\[-4ex]
    % \\
    \< \< \text{ \scriptsize 10.D(E(\colorbox{mygray}{$N_t$}$||S_d||S_p||S_c$,A,p),B,p)} \\[-1ex]
    \< \< \text{ \scriptsize Calculates mod($S_d$,Z)} \\[-1ex]
    \< \< \text{ \scriptsize Calculates mod($S_p$,$W_i$)} \\[-1ex]
    \< \< \text{ \scriptsize Calculates mod($S_c$,$N_c$)} \\[-1ex]
    \< \< \text{ \scriptsize E($N_t+1||ID$,$A_{new}$,q)} \\[-4ex]
    % \\
    }
    \end{adjustwidth}

    The adversary can interact with the protocol the following way:

    1. Adversary waits for a legitimate tag to respond to a reader query(step 3 of the protocol): E($N_t||S$,A,p)
    
    2. Adversary intercepts the response of the reader(step 9 of the protocol)to the tag: E($N_t||S_d||S_p||S_c$,A,p)
    
    3. Having the access to the tag, the adversary corrupts the tag and gets it's internal state. By doing that the adversary now knows A, B, p, f(p), S, Z, $W_i$ and the nonce $N_t$.
    
    4. Adversary decrypts E($N_t||S_d||S_p||S_c$,A,p) using the values B and p. Now the adversary has access to $S_d, S_p, S_c$ and computes the values $A_{new}$ and q.
    
    5. Using the newly obtained $A_{new}$ and q, adversary increments $N_t$, appends ID(found in the tag's memory when the corruption occured) and encrypts.
    
    6. Adversary sends the message to the reader whom can not descern that the tag has been tampered with.

