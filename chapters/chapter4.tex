\chapter{Case study}
\definecolor{mygray}{gray}{0.75}
\definecolor{softred}{RGB}{255, 102, 102}

\section{Initial given protocol attack}

    For the authentication proposed in \cite{BOM}, the tag stores in it's permanent memory the values: \textit{A, B, p, f(p), S, Z, $W_i$}.
    The value $N_c$ designates the cardinality of set f(p), meaning the number of integer divisors of p.
    The global temporary memory will hold the value $N_t$
    used to authenticate the reader. It is generated at step 2 and held until step 10.

    % aici schema pentru interactiunea dintre tag si reader
    \begin{adjustwidth}{-60pt}{}
    \procedureblock[colspace=-0.5cm]{The DBKM-SUEO-SUM-RFID protocol: tag-reader interaction}{
    \textbf{Reader} \< \< \textbf{Tag} \\[-1ex]
    \textit{A, B, p, f(p),$N_r$, S, Z, $W_i$} \< \< \textit{ \colorbox{mygray}{A, B, p, f(p),$N_t$, S, Z, $W_i$, $N_c$}} \\[-1.5ex]
    % \\
    \< \< \text{ \scriptsize 2.Generates \colorbox{olive}{$N_t$}} \\[-1ex]
    \< \< \text{ \scriptsize E($N_t||S$,A,p)} \\[-4ex]
    % \\
    \< \sendmessageleft{top=\text{ \scriptsize 3.E($N_t||S$,A,p)}} \< \\[-4ex]
    % \\
    \text{ \scriptsize 4.D(E($N_t||S$,A,p),B,p)} \< \< \\[-1ex]
    \text{ \scriptsize Generates $N_r$} \< \< \\[-1ex]
    \text{ \scriptsize E($N_r||S$,A,p)} \< \< \\[-1ex]
    % \\
    \text{ \scriptsize ...} \< \< \\[-1ex]
    % \\
    \text{ \scriptsize 8.D(E($N_r||S_d||S_p||S_c$,A,p),B,p)} \< \< \\[-1ex]
    \text{ \scriptsize E($N_t||S_d||S_p||S_c$, A, p)} \< \< \\[-4ex]
    % \\
    \<\sendmessageright{top = \text{ \scriptsize 9.E($N_t||S_d||S_p||S_c$,A,p)}}\<\\[-4ex]
    % \\
    \< \< \text{ \scriptsize 10.D(E($N_t||S_d||S_p||S_c$,A,p),B,p)} \\[-1ex]
    \< \< \text{ \scriptsize Calculates mod($S_d$,Z)} \\[-1ex]
    \< \< \text{ \scriptsize Calculates mod($S_p$,$W_i$)} \\[-1ex]
    \< \< \text{ \scriptsize Calculates mod($S_c$,$N_c$)} \\[-1ex]
    \< \< \text{ \scriptsize E($N_t+1||ID$,$A_{new}$,q)} \\[-4ex]
    % \\
    }
    \end{adjustwidth}

    The adversary can interact with the protocol the following way:

    1. Adversary waits for a legitimate tag to respond to a reader query(step 3 of the protocol): E($N_t||S$,A,p).
    This can be achieved by also sending a query to the tag, compeling it to compute a nonce and sending E($N_t||S$,A,p).
    
    2. Adversary intercepts the response of the reader(step 9 of the protocol)to the tag: E($N_t||S_d||S_p||S_c$,A,p)
    
    3. Having the access to the tag, the adversary corrupts the tag and gets it's internal state. By doing that the adversary now knows A, B, p, f(p), S, Z, $W_i$ and the nonce $N_t$.
    
    4. Adversary decrypts E($N_t||S_d||S_p||S_c$,A,p) using the values B and p. Now the adversary has access to $S_d, S_p, S_c$ and computes the values $A_{new}$ and q.
    
    5. Using the newly obtained $A_{new}$ and q, adversary increments $N_t$, appends ID(found in the tag's memory when the corruption occured) and encrypts.
    
    6. Adversary sends the message to the reader whom can not descern that the tag has been tampered with.

    % aici schema pentru atacul de mai sus

    \begin{adjustwidth}{-60pt}{}
    \procedureblock[colspace=-0.5cm]{The DBKM-SUEO-SUM-RFID protocol: Corrupt attack}{
    \textbf{Reader} \< \< \textbf{Tag} \\[-1ex]
    \textit{A, B, p, f(p),$N_r$, S, Z, $W_i$} \< \< \textit{A, B, p, f(p),$N_t$, S, Z, $W_i$, $N_c$} \\[-1.5ex]
    % \\
    \< \sendmessageright{top=\text{ \scriptsize \colorbox{softred}{$A$ sends query}}} \< \\[-4ex]
    % \\
    \< \< \text{ \scriptsize 2.Generates $N_t$} \\[-1ex]
    \< \< \text{ \scriptsize E($N_t||S$,A,p)} \\[-4ex]
    % \\
    \< \sendmessageleft{top=\text{ \scriptsize 3.E($N_t||S$,A,p)}} \< \\[-4ex]
    % \\
    \text{ \scriptsize 4.D(E($N_t||S$,A,p),B,p)} \< \< \\[-1ex]
    \text{ \scriptsize Generates $N_r$} \< \< \\[-1ex]
    \text{ \scriptsize E($N_r||S$,A,p)} \< \< \\[-1ex]
    % \\
    \text{ \scriptsize ...} \< \< \\[-1ex]
    % \\
    \text{ \scriptsize 8.D(E($N_r||S_d||S_p||S_c$,A,p),B,p)} \< \< \\[-1ex]
    \text{ \scriptsize E($N_t||S_d||S_p||S_c$, A, p)} \< \< \\[-4ex]
    % \\
    \<\sendmessageright{top = \text{ \scriptsize \colorbox{softred}{9.E($N_t||S_d||S_p||S_c$,A,p)}}, bottom = \text{\scriptsize \colorbox{softred}{Intercepted by $A$}}}\<\\[-4ex]
    % \\
    \< \< \text{ \scriptsize \colorbox{softred}{$A$ corrupts the tag, gets B, p and $S$}} \\[-1ex]
    \< \< \text{ \scriptsize \colorbox{softred}{$A$ can decrypt and use the tag}} \\[-1ex]
    \< \< \text{ \scriptsize 10.D(E($N_t||S_d||S_p||S_c$,A,p),B,p)} \\[-1ex]
    \< \< \text{ \scriptsize Calculates mod($S_d$,Z)} \\[-1ex]
    \< \< \text{ \scriptsize Calculates mod($S_p$,$W_i$)} \\[-1ex]
    \< \< \text{ \scriptsize Calculates mod($S_c$,$N_c$)} \\[-1ex]
    \< \< \text{ \scriptsize E($N_t+1||ID$,$A_{new}$,q)} \\[-4ex]
    % \\
    }
    \end{adjustwidth}

\section{Initial given scheme security}
\begin{table}[H]
    \centering
    \caption{The security properties claimed by \cite{BOM}}
    \begin{tabular}{| c | c |}
        \hline
        Protocol & Claimed \\
        \hline
        Mutual authentication & Yes \\
        Location tracking & Yes \\
        DoS & Yes \\
        Impersonation attack & Yes  \\
        Man-in-the-middle attack & Yes  \\
        Replay attack & Yes  \\
        De-synchronization & Yes  \\
        Forward secrecy & Yes \\
        \hline
    \end{tabular}
\end{table}

    However many of the security claims do not hold in the context of tag corruption. An important note is that an adversary can corrupt
    a tag during the execution of the protocol, being it earlier or later in its computation.

    In the case of \textbf{mutual authentication}, an adversary can use the Corrupt() oracle to obtain secret $S$. By calling the CreateTag()
    oracle and using value $S$ illegitimate tags can be authenticated by a reader. Thus tag authentication is countered.
    For reader authentication the adversary can corrupt
    multiple tags and store combinations of matrices $A$, $B$ and modulus $p$ and with an illegitimate reader be able to succesfully
    authenticate legitimate tags. This is possible by being able to decrypt (step 4 of the protocol) and obtain nonce $N_t$ which can
    be returned to the tag (step 9 of the protocol). This way of thwarting authentication is also the base for \textbf{impersonation attacks}.
    
    In the case of \textbf{location tracking} an adversary can corrupt a tag and infer relations or straight out obtain secret value $S$, meaning
    they can link it with a particular location. Strong adversaries can corrupt the tags and release them back. By sending a query(step 1
    of the protocol) and breaking reader authentication the adversary can further track the tag in case it was moved.

    A \textbf{DoS attack} consists in sending large amounts of data to a system in order to cause it to malfunction. The original protocol
    stops its execution as soon as a party fails to authenticate another, in an effort to prevent such an attack. However if the adversary 
    has access to sensitive data from tags then a DoS attack is more plausible than initially thought.

    The adversary can also use the sensitive information from a tag to implement a \textbf{man-in-the-middle attack}, for example the 
    interaction described in section 4.1.

\section{Achievable classes of privacy}

    Clearly access to the Corrupt oracle has major consequences on RFID schemes. Following the impossibility results from section 3, 
    depending on the tags used, these classes of privacy can be achieved:

    \begin{table}[H]
    \centering
    \caption{The security properties achievable for various tags}
    \begin{tabular}{| c | c |}
        \hline
        Class of tag & achievable privacy \\
        \hline
        With temporary state disclosure & weak \& narrow-weak privacy\\
        Without temporary state disclosure & destructive \& narrow-destructive privacy\\
        Resetable & no privacy \\
        Stateless & narrow-destructive \& forward privacy \\
        \hline
    \end{tabular}
    \end{table}

    These results lead to an extended discusion about security for a given RFID scheme. The \cite{BOM} scheme achieves mutual
    authentication between tags and reader and between reader and server. For the former, the reader authenticates the tag by
    receiving the secret value $S$ and querying it into its internal database to check its legitimacy. Reader authentication is
    achieved by decrypting the message at step 9 and receiving back value $N_t$. Only the reader should be able to decrypt 
    E($N_t||S$,A,p) (step 3) and obtain $N_t$ and be able to send it back, thus proving authenticity.
    
    % aici schema pentru mutual authentication
    \begin{adjustwidth}{-60pt}{}
    \procedureblock[colspace=-0.5cm]{The two way DBKM-SUEO-SUM-RFID protocol: use of temporary variable $N_t$}{
    \textbf{Reader} \< \< \textbf{Tag} \\[-1ex]
    \textit{A, B, p, f(p),$N_r$, S, Z, $W_i$} \< \< \textit{A, B, p, f(p),$N_t$, S, Z, $W_i$, $N_c$} \\[-1.5ex]
    % \\
    \< \< \text{ \scriptsize 2.Generates \colorbox{mygray}{$N_t$}} \\[-1ex]
    \< \< \text{ \scriptsize E($N_t||S$,A,p)} \\[-4ex]
    % \\
    \< \sendmessageleft{top=\text{ \scriptsize 3.E($N_t||S$,A,p)}} \< \\[-4ex]
    % \\
    \text{ \scriptsize 4.D(E(\colorbox{mygray}{$N_t$}$||$\colorbox{olive}{$S$},A,p),B,p)} \< \< \\[-1ex]
    \text{ \scriptsize Generates $N_r$} \< \< \\[-1ex]
    \text{ \scriptsize E($N_r||S$,A,p)} \< \< \\[-1ex]
    % \\
    \text{ \scriptsize ...} \< \< \\[-1ex]
    % \\
    \text{ \scriptsize 8.D(E($N_r||S_d||S_p||S_c$,A,p),B,p)} \< \< \\[-1ex]
    \text{ \scriptsize E(\colorbox{mygray}{$N_t$}$||S_d||S_p||S_c$, A, p)} \< \< \\[-4ex]
    % \\
    \<\sendmessageright{top = \text{ \scriptsize 9.E($N_t||S_d||S_p||S_c$,A,p)}}\<\\[-4ex]
    % \\
    \< \< \text{ \scriptsize 10.D(E(\colorbox{mygray}{$N_t$}$||S_d||S_p||S_c$,A,p),B,p)} \\[-1ex]
    \< \< \text{ \scriptsize Calculates mod($S_d$,Z)} \\[-1ex]
    \< \< \text{ \scriptsize Calculates mod($S_p$,$W_i$)} \\[-1ex]
    \< \< \text{ \scriptsize Calculates mod($S_c$,$N_c$)} \\[-1ex]
    \< \< \text{ \scriptsize E($N_t+1||ID$,$A_{new}$,q)} \\[-4ex]
    % \\
    }
    \end{adjustwidth}
    
    For tag-reader mutual authentication recall:

    \textbf{Theorem 1}: \textit{There is no RFID system in Vaudenay's model that achieves both reader authentication and narrow-forward privacy 
    under temporary state disclosure.}

    This means that for the scheme presented the achievable privacy is weak or narrow-weak for the case in which the adversary has no
    access to the Result oracle. This result is based on the presumption that temporary state disclosure includes the temporary
    memory of the tag and additionally the persistent state.

    For the case of disclosure of only the permanent state of the tag the impossibility results of theorem 2 state:

    \textbf{Theorem 2}: \textit{There is no RFID system in Vaudenay's model that achieves both reader authentication and narrow-strong privacy 
    under permanent state disclosure.}

    Thus under the presumption that the Corrupt oracle yields only the permanent state the scheme can achieve at most destructive or
    narrow-destructive privacy.

    Resetable tags and the ability of the adversary to change the state to its initial values lead to the impossibility of narrow-weak
    privacy. Thus the use of such tags does not guarantee even the weakest class of privacy.

    Stateless tags do not change their internal memory and the use of them can yield at best narrow-destructive or forward privacy.
