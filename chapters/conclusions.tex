\chapter*{Conclusions} 
\addcontentsline{toc}{chapter}{Conclusions}

% The use of temporary variables needs to be done with great care, especially when the desired 
% properties for a specific application are both security and privacy. 

% The specific application in which tags are used derive also the need for specific security properties
% and for the case of privacy the restrictions of tags have a major role.

% The need for temporary variables arose from achieving authentication and in the effort of achieving security while privacy took a secondary role.
% However for different classes of tags and their associated capabilities are linked different classes of adversaries and thus different potential achievable privacy.

% What do I want to say:
% - there needs to be special attention payed to privacy where it is desired
% - protocols often focus on security and privacy comes second
% - the tags are deeply restrained devices and these constraints often define the achievable privacy
% - we proved that even 

RFID schemes have vast applications that come each with various desired properties, being for example security in the sense of resistence to impersonation attacks 
or privacy in the sense of linking tags and location tracking. If the application requires privacy there needs to be special attention payed to information leaks
and how an adversary with the ability to corrupt tags can thwart this property.

RFID schemes have the role first and foremost to succesfully identify a tag, thus security assumed a central role, however in the present environment focused on 
gathering data privacy occupies a key position.

The tags are deeply restrained devices and these constraints often define the achievable privacy. These challenges define the difficulties in developing protocols
resistent to attacks.