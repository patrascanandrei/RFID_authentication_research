\chapter*{Conclusions} 
\addcontentsline{toc}{chapter}{Conclusions}

\section*{Future directions and research}
\addcontentsline{toc}{section}{Future directions and research}
The Corrupt oracle has vast implications to the security and privacy of RFID schemes. We showed the results that are achievable for tags
when the adversary can obtain the full state or just the persistent state and the properties achievable for resettable and stateless tags.
A reasonable direction for RFID schemes is what can be realised when tamper-proof tags are used, i.e. PUFs. A PUF is a tag that irreversibly changes 
its response if there are attempts to physically access it. Using this \cite{PUFs} has achieved a higher level of privacy then it was 
previously achievable.

\section*{Closing remarks}
\addcontentsline{toc}{section}{Closing remarks}
RFID schemes have vast applications that come each with various desired properties, being for example security in the sense of resistance to impersonation attacks 
or privacy in the sense of linking tags and location tracking. If the application requires privacy there needs to be special attention payed to information leaks
and how an adversary with the ability to corrupt tags can thwart this property.

The tags are deeply restrained devices and these constraints often define the achievable privacy. These challenges define the difficulties in developing protocols
resistant to attacks.

RFID schemes have the role first and foremost to successfully identify a tag, thus security assumed a central role, however in the present environment focused on 
gathering data privacy also occupies a key position.