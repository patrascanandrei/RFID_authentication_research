\chapter{Security and Privacy(Vaudenay's RFID model)}

\section{RFID Schemes}

    A typical RFID system includes three components: a coil(or antenna), a transceiver(with a decoder) and a
    transponder(a radio-frequency tag) embedded with unique information. The antenna emits radio signals in order to activate the tag and to read or write 
    data on it. Antennas establish the communication between the transceiver and the tags. They can be packaged with the transceiver and the decoder in order to become
    a reader. 

\vspace*{0.1cm} 

        \textbf{The tags} are passive transponders identified by a unique ID. Their technical parameters often are: memory, power supply, shape, size and the presence of a 
    microprocessor. For many applications the tags need to be very small and cheap so their attributes are deeply constrained. A tag most often:
    \begin{itemize}
        \item is passive: because it has no batteries it can operate only when queried by the reader and can only respond for a short time after;
        \item has limited memory: a few kilobits of memory;
        \item has limited computational ability, can only perform basic calculations: hash calculations \cite{Feldhofer}, pseudorandom generation \cite{Robshaw}, 
        symmetric encryption \cite{Feldhofer2}. Some elliptic-curve arithmetic and public-key cryptography may fit, but remain expensive so far;
        \item provides no physical security: each tag can be opened(corrupted) and thus revealing its memory;
        \item communicates at up to a fixed distance: the tag has a range of a few meters.
    \end{itemize}
    
        In RFID Moore's law cannot be directly applied. In fact, the main goal of RFID transponders producers is to keep prices down and so the performance takes
    a secondary, less important role.

\vspace*{0.1cm} 
        
        \textbf{The reader} is composed by one or more transceivers and a backend processing system(sometimes in the literature the reader denotes just the transceiver). 
    The reader is able to accomodate multiple tags at the same time although a high number of them can lead to substantial delays for authentication. 
    
\vspace*{0.1cm}

        \textbf{The server} is the component that handles the complex cryptographic calculations and eases the load on readers whenever possible. The traditional protocols
    included fixed reader positions and assumed secure(wired) communication. However in practical applications the communication between server and reader is also done via 
    Radio-Frequency Signals.

\vspace*{0.1cm}

        RFID protocols are used most often for two goals: to identify tags(by recovering their unique ID) and to authenticate tags(to make sure a tag is legitimate i.e. registered in 
    the database). 

\vspace*{0.2cm}

    A frequently used and accepted model for security and privacy for RFID schemes is Vaudenay's
    model, proposed in \cite{Vaudenay}. 

    \textit{RFID schemes}: A RFID scheme is formed by a triple:

    - a setup scheme for the reader: SetupReader($\lambda$): generates a pair of keys: $K_P$ and 
    $K_S$ for the reader depending on the security paremeter $\lambda$. The public key is released
    to the public and the secret key is stored on the reader database.

    -a setup scheme for the tag: SetupTag(ID) $\rightarrow$ (K, S): K is the tag secret and S denotes the 
    initial state of the tag. Furthermore if the tag is legitimate the values ID and K are stored
    on the memory of the reader.
    
    -an interactive protocol between the reader and the tag, at the end of comunication the reader
    will output a value ($\bot$, correct ID, incorrect ID) indicating succesfull authentication or 
    not. Formally:
    $Ident[\mathcal{T}_{ID}:S; \mathcal{R}: sk_{\mathcal{R}}, DB; *:pk_{\mathcal{R}}] \rightarrow 
    [\mathcal{T}_{ID}:out_{\mathcal{T}_{ID}}; \mathcal{R}:out_{\mathcal{R}}]$

\vspace{0.3cm}
\hspace{1cm}
\section{Adversaries and oracles}
\hspace{0.5cm}
    \textit{Adversaries:} An adversary is often defined by what action they can perform, they are 
    known as oracles. A tag can be either drawn or free, this means that it is in reach of the 
    adversary or not. A virtual tag is a temporary identifier for a tag when it is drawn.

    \vspace{0.3cm}
    
    Below are the 8 oracles defined in Vaudenay's model:

    1. $CreateTag^b$(ID): created a new tag, legitimate(b=1) or not(b=0) with the unique ID.
        This oracles uses SetupTag(ID) and if the tag is legitimate then the reader 
        updates it's database.
    
    2. DrawTag(distr) $-> (vtag_1, b_1, ... , vtag_n, b_n$): selects at random n tags following the 
    distribution distr and draws them to be used by the adversary. The oracle provides
    an array of virtual identifiers ($vtag_1, ... ,vtag_n$) as the adversary needs to reference them.
    and the bits $b_k$ meaning authentic or not. A hidden table $\Gamma$ is kept by the oracle
    containing the association between the temporary identifiers and the real IDs.

    3. Free(vtag): the vtag is eliminated from the set the attacker works with.
    
    4. Launch $-> \pi$: the reader launches a new protocol instance $\pi$.

    5. SendReader($m, \pi) -> (m'$): sends the reader a message m and receives as response 
    the message $m'$ as part of the protocol $\pi$.

    6. SendTag($m, \pi) -> (m'$): sends the tag a message m and receives as response 
    the message $m'$. If m is empty the response is the first message transmitted by the tag for
    a given protocol. 

    7. Result($\pi$): After the protocol has ended on the reader side, output 1 if the tag was 
    authenticated and 0 otherwise.

    8. Corrupt(vtag): Ouputs the internal/permanent state of the tag named vtag.

    \vspace{0.3cm}
    \section{Classes of adversaries}
    \hspace{0.5cm}
    Depending on how these oracles are used/the access the attacker has, 4 kinds of adversary are defined:
    
    1. STRONG: class of adversary who have access to all the oracles, no limitations

    2. DESTRUCTIVE: adversaries who never use a tag after a Corrupt(vtag) oracle is called
        i.e. the tag is destroyed

    3. FORWARD: adversaries that once they call Corrupt(vtag) oracles can only call the same oracle
        i.e the corruption of tags are left at the end of the attacking phase

    4. WEAK: adversaries who have no access to the Corrupt(vtag) oracle, the internal memory of the tags remains a secret. 

    Another way to consider adversaries is by their access to the Result($\pi$) oracle. If they have the access, they are considered to be
        \textit{wide} adversaries, if not, they are called \textit{narrow}.

    \section{Security}

    \hspace*{0.4cm} 
    The security definition of the Paise-Vaudenay model\cite{PV model} encompases attacks where the adversary aims to impersonate or forge a legitimate tag 
    $\mathcal{T}$ or the reader $\mathcal{R}$. 

    \textit{Reader authentication}: The definition of reader authentication is based on a security experiment $Exp_{\mathcal{A}_{sec}}^{\mathcal{R}-aut}$
    where a strong adversary must impersonate $\mathcal{R}$ to a legitimate tag $\mathcal{T}_{ID}$. To exclude trivial attacks, the attacker $\mathcal{A}_{sec}$
    must compute some of the protocol messages, not just forward them from $\mathcal{R}$ to $\mathcal{T}_{ID}$. $Exp_{\mathcal{A}_{sec}}^{\mathcal{R}-aut}$ = 1
    denotes that $\mathcal{A}_{sec}$ wins the security experiment.

    \textbf{Definition 1}: An RFID system achieves reader authentication if for every strong adversary $\mathcal{A}_{sec}$ Pr[$Exp_{\mathcal{A}_{sec}}^{\mathcal{R}-aut}$ = 1]
    is negligible.

    % Consider an adversary in the class STRONG. They win if there is an instance of the protocol run
    % by the reader where a legitimate tag is authenticated but done so when the specific steps of the 
    % protocol are followed out of order(interleaved). If the probability of the adversary to win is 
    % negligible then a RFID scheme is considered secure.

    % For a mutual authentication scheme the same property needs to be respected for the tag to authenticate
    % the reader.
    \section{Privacy}
    \hspace*{0.4cm} 

    Vaudenay's notion of privacy captures anonymity and unlinkability and focuses on the privacy loss in the wireless link. It is based on 
    the existence of a simulator $\mathcal{B}$, called a \textbf{blinder}, that can simulate the reader and the tags without knowing their secrets.
    
    The simulation aspect of the blinder catches the case of an adversary that can't use the messages transmitted, they only know how the protocol
    interacts and the actual data traffic is simulated. By contrasting this attacker with an adversary that eavesdropes, denies or changes messages
    several conclusions can be drawn about the properties of a protocol.

    \vspace{0.3cm}

    The privacy definition can be formalized by a privacy experiment $Exp_{\mathcal{A}_{prv}}^{prv-b} = b^{'}$ defined by an adversary of class P, a
    security parameter s and $\mathcal{b} \in_{R}$ {0,1}. In the first phase of the experiment, the reader is initialized (using SetupReader($1^s$)). 
    The public key is available to $\mathcal{A}_{prv}$ and $\mathcal{B}$. $\mathcal{A}_{prv}$ interacts with the protocol according to its constraints.
    If b = 1, all queries to Launch, SendReader, SendTag and Result oracles are handled by $\mathcal{B}$. Moreover, the blinder can surveil all the other
    oracles. In phase two, the adversary cannot call the oracles any longer but is given the hidden table $\Gamma$ of the DrawTag oracle. Finally, 
    $\mathcal{A}_{prv}$ returns bit $b^{'}$.

    Informally, $Pr[Exp_{\mathcal{A}_{prv}}^{prv-b}]$ denotes the probability that an adversary $\mathcal{A}_{prv}$
    can guess the bit b, meaning whether they interact with the real oracles or through a blinder $\mathcal{B}$.

    \vspace{0.3cm}

    \textbf{Definition 2}: Let P be one of the adversary classes specified above. An RFID system is P-private if for
    every adversary $\mathcal{A}_{prv}$ of P there exists a probabilistic polynomial time algorithm $\mathcal{B}$ 
    (blinder) such that the advantage $Adv_{\mathcal{A}_{prv}}^{prv} = | Pr[Exp_{\mathcal{A}_{prv}}^{prv-0} = 1] - Pr[Exp_{\mathcal{A}_{prv}}^{prv-1} = 1] |$ 
    of $\mathcal{A}_{prv}$ is negligible. $\mathcal{B}$ simulates the oracles: Launch, SendReader, SendTag and
    Result to $\mathcal{A}_{prv}$ without having access to the reader's secret key or database. It is considered the all
    oracle queries $\mathcal{A}_{prv}$ makes and the responses received are also available to $\mathcal{B}$.

    Privacy is defined by the ability of the adversary to infer relations about the ID from the protocol
    messages. If a scheme is resistent to all P-private adversaries(belonging to P class, with all
    P specific access) then the scheme achieves P-privacy.

    An adversary $\mathcal{A}$ is trivial if there is a negligible difference between the chances
    of an adversary that interacts with the system to win and the chances of a blinded adversary.

    All privacy notions defined in the Paise-Vaudenay Model are linked the following way:

    \[
    \begin{array}{ccccccc}
        \text{Strong} & \Rightarrow & \text{Destructive} & \Rightarrow & \text{Forward} & \Rightarrow & \text{Weak} \\
        \Downarrow & &  \Downarrow  & &\Downarrow  & & \Downarrow \\
        \text{Narrow-Strong} & \Rightarrow & \text{Narrow-Destructive} & \Rightarrow & \text{Narrow-Forward} & \Rightarrow & \text{Narrow-Weak}
    \end{array}
    \]
